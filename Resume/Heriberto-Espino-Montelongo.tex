\documentclass[9pt]{extarticle} % Tamaño de fuente base 12pt
\usepackage{hyperref} % Para enlaces
\usepackage{enumitem} % Para listas personalizadas
\usepackage[utf8]{inputenc} % Codificación UTF-8 (necesario en algunos sistemas)
\usepackage[T1]{fontenc} % Codificación de fuente T1
\usepackage[english]{babel} % Lenguaje en inglés
\usepackage[letterpaper, left=2.5cm, top=1cm, right=2.5cm, bottom=0.49cm]{geometry} % Márgenes personalizados
\usepackage{helvet} % Fuente Helvetica
\renewcommand{\familydefault}{\sfdefault} % Usa Helvetica como fuente por defecto
\usepackage{setspace} % Paquete para interlineado
\usepackage{lipsum} % Para texto de ejemplo
\usepackage{xcolor}

% Configurar interlineado
\setstretch{1.2} % Establece el interlineado a 1.2 (más espacioso)

% Configuración de párrafos
\setlength{\parindent}{0pt} % Sin sangría al inicio de los párrafos





\definecolor{customblue}{HTML}{001c7f}




\begin{document}

% Cambia el tamaño de la fuente a 10pt antes de tu nombre
{\fontsize{10pt}{12pt}\selectfont
\begin{center}
    \textbf{Heriberto Espino Montelongo} (English Version) \\
    \vspace{-1ex} % Space reduction
\end{center}
}

\begin{center}
    Puebla, Mexico, 72160 \textbullet \ 
    \href{mailto:heriberto.espinomo@udlap.mx}{\textcolor{customblue}{\underline{heriberto.espinomo@udlap.mx}}} \textbullet \ 
    \href{tel:+522228101202}{\textcolor{customblue}{\underline{222 810 1202}}} \textbullet \ 
    \href{https://github.com/heritaco}{\textcolor{customblue}{\underline{GitHub: heritaco}}}
\end{center}



\begin{center}
    \vspace{1ex}
    \textbf{Education}
    \vspace{-2ex}
\end{center}

\textbf{Universidad de las Américas Puebla} \hfill Puebla, Mexico\\
Bachelor's Degree in Actuarial Science, GPA: 9.5/10. \hfill  44 of 50 courses completed (2021 -- 2025) \\
Relevant Coursework: Investment Portfolios, Derivative Products, Statistical Modeling and Regression Analysis, Time Series Analysis, Risk Theory, Measure Theory, Demography.\\
\\
\textbf{Universidad de las Américas Puebla} \hfill Puebla, Mexico\\
Bachelor's Degree in Data Science, GPA: 9.6/10. \hfill  44 of 50 courses completed (2021 -- 2025)\\
Relevant Coursework: Advanced Optimization, Pattern Recognition, Artificial Intelligence, Data Mining, Topological Data Analysis, Geospatial Data Analysis.












\begin{center}
    \vspace{1ex}
    \textbf{Projects}
    \vspace{-1ex}
\end{center}

\textbf{Predictive Modeling, Machine Learning and Portfolio Optimization}  \hfill Puebla, México\\
Archieving the optimum portfolios for Spot and CFDs to different areas\hfill 2024-2025
\begin{itemize}[noitemsep, topsep=0pt, partopsep=0pt, parsep=0pt]
    \item Optimized portfolios using Markowitz and Black-Litterman models obtaining the best Sharpe ratios, Sortino ratios, Calmar ratios, and Treynor ratios.
    \item Simulated market behavior using Geometric Brownian Motion for traditional stocks and advanced stochastic processes (Constant Elasticity Variance, Variance-Gamma) for high-volatility assets. 
    \item Simulated prices movements with SARMIA-GARCH models.
    \item Built a Forsets models by bagging, boosting and stacking for accurate classification and prediction of prices.
\end{itemize}  
More projects at \href{https://github.com/heritaco}{\textcolor{customblue}{\underline{GitHub}}}.











\begin{center}
    \vspace{1ex}
    \textbf{Activities}
    \vspace{-1ex}
\end{center}

\textbf{The William Lowell Putnam Mathematical Competition} \hfill North America\\
\textbf{Competitor} \hfill December 2024\\
Competed in the most prestigious undergraduate mathematics competitions at North America level, solving  problems in areas such as Number Theory, Algebra, Combinatorics, and Geometry.\\

\textbf{Olimpiada de Matemáticas UDLAP} \hfill Puebla, Mexico\\
\textbf{Participant} \hfill 2024, 2025\\
Top 10 among all university competitors from all areas, covering topics such as Algebra, Linear Algebra, Combinatorics, Statistics, Analytic Geometry, Set Theory, and Calculus.\\

\textbf{Reto Actinver 2024} \hfill Mexico\\
\textbf{Participant} \hfill September 2024\\
Participated in a national financial challenge, focused on investment strategies. Analyzed market data, applied quantitative methods, and developed strategies to maximize returns while managing risk in a simulated investment environment.\\


\textbf{Reto Coppel} \hfill Puebla, Mexico\\
\textbf{Participant} \hfill March 2025\\
Applied Spectral Clustering to identify underperforming stores and used Queueing Theory to optimize service for improved operational efficiency.














\begin{center}
    \vspace{1ex}
    \textbf{Technical Skills}
    \vspace{-1ex}
\end{center}

\textbf{Programming Languages:} Python, R, C, C++, SQL, Java, Mosel.\\
\textbf{Office Tools:} Microsoft Excel (advanced, including VBA), Word, PowerPoint, Power BI.\\
\textbf{Database Management Systems:} MySQL.\\
\textbf{Version Control:} Git, GitHub.\\
\textbf{Scripting and Shell Environments:} Bash (Ubuntu/Linux Terminal), PowerShell (Windows 11).\\
\textbf{Markup and Document Languages:} HTML, LaTeX, Markdown.\\























































\newpage

% Cambia el tamaño de la fuente a 10pt antes de tu nombre  
{\fontsize{10pt}{12pt}\selectfont  
\begin{center}  
    \textbf{Heriberto Espino Montelongo} (Versión en español)\\  
    \vspace{-1ex} % Reducción de espacio  
\end{center}  
}

\begin{center}  
    Puebla, México, 72160 \textbullet \   
    \href{mailto:heriberto.espinomo@udlap.mx}{\textcolor{customblue}{\underline{heriberto.espinomo@udlap.mx}}} \textbullet \   
    \href{tel:+522228101202}{\textcolor{customblue}{\underline{222 810 1202}}} \textbullet \   
    \href{https://github.com/heritaco}{\textcolor{customblue}{\underline{GitHub: heritaco}}}  
\end{center}

\begin{center}  
    \vspace{1ex}  
    \textbf{Formación Académica}  
    \vspace{-2ex}  
\end{center}

\textbf{Universidad de las Américas Puebla} \hfill Puebla, México\\  
Licenciatura en Ciencia Actuarial, promedio: 9.5/10. \hfill 44 de 50 cursos completados (2021 -- 2025) \\  
Cursos relevantes: Portafolios de Inversión, Productos Derivados, Modelado Estadístico y Análisis de Regresión, Análisis de Series Temporales, Teoría del Riesgo, Teoría de la Medida, Demografía.\\  
\\  
\textbf{Universidad de las Américas Puebla} \hfill Puebla, México\\  
Licenciatura en Ciencia de Datos, promedio: 9.6/10. \hfill 44 de 50 cursos completados (2021 -- 2025)\\  
Cursos relevantes: Optimización Avanzada, Reconocimiento de Patrones, Inteligencia Artificial, Minería de Datos, Análisis Topológico de Datos, Análisis Geoespacial de Datos.

\begin{center}  
    \vspace{1ex}  
    \textbf{Proyectos}  
    \vspace{-1ex}  
\end{center}

\textbf{Modelado Predictivo, Aprendizaje Automático y Optimización de Portafolios} \hfill Puebla, México\\  
Obtención de portafolios óptimos para Spot y CFDs en diferentes áreas \hfill 2024-2025  
\begin{itemize}[noitemsep, topsep=0pt, partopsep=0pt, parsep=0pt]  
    \item Optimización de portafolios usando modelos de Markowitz y Black-Litterman, obteniendo las mejores razones de Sharpe, Sortino, Calmar y Treynor.  
    \item Simulación del comportamiento del mercado mediante Movimiento Browniano Geométrico para acciones tradicionales y procesos estocásticos avanzados (Elasticidad Constante de Varianza, Variance-Gamma) para activos de alta volatilidad.  
    \item Simulación de movimientos de precios con modelos SARMIA-GARCH.  
    \item Construcción de modelos Random Forest mediante técnicas de bagging, boosting y stacking para clasificación precisa y predicción de precios.    
\end{itemize}  
Más proyectos en \href{https://github.com/heritaco}{\textcolor{customblue}{\underline{GitHub}}}.

\begin{center}  
    \vspace{1ex}  
    \textbf{Actividades}  
    \vspace{-1ex}  
\end{center}

\textbf{Concurso Matemático William Lowell Putnam} \hfill Norteamérica\\  
\textbf{Competidor} \hfill Diciembre 2024\\  
Participé en la competencia universitaria de matemáticas más prestigiosa a nivel de Norteamérica, resolviendo problemas en áreas como Teoría de Números, Álgebra, Combinatoria y Geometría.\\

\textbf{Olimpiada de Matemáticas UDLAP} \hfill Puebla, México\\  
\textbf{Participante} \hfill 2024, 2025\\  
Top 10 entre todos los competidores universitarios de diversas áreas, cubriendo temas como Álgebra, Álgebra Lineal, Combinatoria, Estadística, Geometría Analítica, Teoría de Conjuntos y Cálculo.\\

\textbf{Reto Actinver 2024} \hfill México\\  
\textbf{Participante} \hfill Septiembre 2024\\  
Participé en un reto financiero nacional enfocado en estrategias de inversión. Analicé datos de mercado, apliqué métodos cuantitativos y desarrollé estrategias para maximizar retornos y gestionar riesgos en un entorno simulado de inversión.\\

\textbf{Reto Coppel} \hfill Puebla, México\\  
\textbf{Participante} \hfill Marzo 2025\\  
Apliqué clustering espectral para identificar sucursales con bajo rendimiento y utilicé teoría de colas para optimizar el servicio y mejorar la eficiencia operativa.

\begin{center}  
    \vspace{1ex}  
    \textbf{Habilidades Técnicas}  
    \vspace{-1ex}  
\end{center}

\textbf{Lenguajes de Programación:} Python, R, C, C++, SQL, Java, Mosel.\\  
\textbf{Herramientas de Oficina:} Microsoft Excel (avanzado, incluyendo VBA), Word, PowerPoint, Power BI.\\  
\textbf{Sistemas de Gestión de Bases de Datos:} MySQL.\\  
\textbf{Control de Versiones:} Git, GitHub.\\  
\textbf{Entornos de Scripting y Consola:} Bash (Terminal Ubuntu/Linux), PowerShell (Windows 11).\\  
\textbf{Lenguajes de Marcado y Documentación:} HTML, LaTeX, Markdown.\\

\end{document}



















\end{document}